\chapter{Style} \label{ch:Style}
This chapter provides some information and tips about writing style. 
As TUM student, you can benefit from free consulting of the English Writing Center\footnote{\url{https://www.sprachenzentrum.tum.de/sprachen/englisch/english-writing-center/}}, which are highly recommended so that your scientific advisor can focus her/his support on the content of your work (rather than having to polish language first).

\section{General Information}
Keep it simple. Your reader should be able to reach the end of a sentence and remember how it began. If your sentences are long and complicated, see if you can break them up. 

\section{Punctuation}
English punctuation can be tricky. Many rules are steadfast -- there are no exceptions. Others, however, vary based on which style manual one uses. In general, there are two preferred styles for technical fields like computer science: that of the \textbf{Institute of Electrical Engineers (IEEE)} and that of the \textbf{Association for Computing Machinery (ACM)}. You can find more information about their citation formatting on their respective websites or in Section \ref{sec:Citation}. The \textbf{IEEE} style falls under the broader Chicago Manual of Style, so the answers to punctuation and syntax questions can be found there. Note that these all use American formatting standards.

\subsection{Commas}
Some typical rules regarding commas include:
\begin{enumerate}
	\item You \textbf{must} put a comma between two \textbf{independent clauses} (i.e., complete sentences) which are joined by a conjunction, \textit{for, and, nor, but, or, yet, so}.
	\begin{itemize}
		\item The exception to this rule is very short sentences -- \textit{I computed A and my partner computed B}. It is unlikely that you will encounter such sentences in your writing, so this exception can be ignored 95\% of the time.
	\end{itemize}
	\item To avoid confusion, use the serial/Oxford comma, i.e., the final comma in a list with three or more items.	
	%\item You \textbf{may} omit the final comma in a list with three or more items; \textbf{however}, IEEE (and by extension, Chicago) Style typically requires the use of the serial/Oxford comma.
	\begin{itemize}
		\item For example: We present our method, our results, and a discussion.
	\end{itemize}
%	\begin{itemize}
%		\item My \$ 10 million estate is to be split among my husband, daughter, son, and nephew.
%	\end{itemize}
%	Omitting the comma after son would indicate that the son and nephew would have to split one-third of the estate.
\item Never put a comma between \textit{to be} and \textit{that}. Likewise, do not put a comma before any form of \textit{to be}. 
\begin{itemize}
	\item Incorrect: The biggest confusion, is that there is not enough data. \\
	Incorrect: The biggest confusion is, that there is not enough data.\\
	Correct: The biggest confusion is that there is not enough data.
	 \end{itemize}
 *There may be instances when you \textit{should} put a comma before a form of \textit{to be}, but this will serve a different grammatical function, e.g., to set off an appositive. An appositive is a noun or pronoun which is set next to another noun or pronoun in order to explain it. For example: ``The conclusion, the part of the paper which summarizes everything, is the trickiest part to write.''
 \item Put a comma before ``which'' if the text that comes before it can be understood without the text that comes after it. Otherwise, do not.
 \begin{itemize}
 	\item For example: We use the xyz control method, which was presented in Chapter 2.
 \begin{itemize}
 	\item Here, the fact that this method was presented in Chapter 2 is additional information. You can understand the sentence without it.
 \end{itemize}
We use the control method which was presented in Chapter 2.
\begin{itemize}
	\item Here, you need to define which control method you are referring to; thus, that information is vital to properly understanding the sentence. 
\end{itemize}
 \end{itemize}
\item Ue a comma when beginning sentences with introductory words such as ``well'', ``now'', or ``yes''.
\item According to the IEEE guidelines, you should put a comma after ``\textbf{i.e.}'' and ``\textbf{e.g.}'' This is widely accepted by most American style guides. It is less common in British English. 
\item For more comma rules, see the enclosed document \texttt{more\_comma\_rules.pdf}.
\end{enumerate}

\subsection{Semi-Colon}
The semi-colon has two main functions. It can be used to join two complete sentences whose meanings are closely related: ``The results are inconclusive; this is due to the fact that the data does not represent all possibilities.''

Additionally, the semi-colon can be used as a sort of ``super-comma'' when lists have internal punctuation: ``This section has an introduction, which gives an overview; a main part, where the methods are discussed; and a conclusion.'' 


\section{Parts of Speech - Common Errors}
In this section, common errors/confusing rules will be discussed.
\subsection{Adverbs}
What is an adverb, again? An adverb modifies a verb, an adjective, another adverb, or even a phrase. It typically, although not always, ends in ``-ly.''

\subsubsection{Placement}
\underline{Modifying Adjectives}\\

\noindent When adverbs modify an adjective, they should be placed directly before the word they are modifying. For example, ``\textit{Table 2 shows \textbf{similarly promising} results.}''\\

\noindent \underline{Modifying Sentences}\\

\noindent When modifying sentences, adverbs can be placed in four different positions:
\begin{itemize}
	\item at the beginning: \textit{\textbf{Usually}, such experiments are doomed to fail from the start.}
	\item at the end: \textit{We then repeat these steps \textbf{frequently}.}\\
	Adverbs appearing at the end of the sentence include:
	
	\begin{itemize}
		\item adverbs of frequency - \textit{usually}, \textit{normally}, \textit{often}, etc.
		\item adverbs of manner, i.e., how - \textit{slowly}, \textit{badly}
		\item adverbs of time - \textit{today}, \textit{this year}
	\end{itemize}
	\item after \textit{to be} and auxiliary verbs, e.g., \textit{can}, \textit{may}, \textit{will}, etc.: \textit{The problem will \textbf{definitely} be fixed.}
	\item before all other verbs: \textit{We \textbf{quickly} solve the equation.}
\end{itemize}

\subsection{Verbs}
\subsubsection{Tense}
The question of tense can be rather complicated. You may have learned that one should always use the present tense, even when referring to past literature. Different styles have different rules, though often these ``rules'' are simply suggestions. The main body of your text should be written in the present tense. However, your literature review may use different tenses, depending on what you wish to express.

The generally accepted guidelines on using tenses in the literature review are as follows:
\begin{itemize}
	\item Simple Past: Use the simple past tense (\textit{found}, \textit{reported}, \textit{introduced}) when you are describing the results/methods of an already published paper. This tense is used when the citation/source is the subject of the sentence: ``Smith [1] found that this method was not very successful.''
	\item Present: Use the present tense in the literature review if you are offering your own opinion of the literature. For example, ``Smith [1] argues that this approach does not make sense; however, it is clear that the methodology had some flaws.''
	\item Present Perfect: This is typically used to indicate that the research you are citing is still ongoing/relevant, or that your source is quite recent It can also be used to generalize about past literature. - ``Similar findings have been discussed more recently [7], [8], [10].''
\end{itemize}

\subsubsection{Voice}
As with tense, selecting which voice (active or passive) has advocates in both corners. Some writers will tell you to avoid the passive voice at all costs. Others will say that scientific writing uses the passive voice much more often than other disciplines. Overusing the passive voice can get dry though, so it is generally good to vary your sentences.

\begin{itemize}
	\item Active Voice: 
	\begin{itemize}
		\item ``We show the results in Table 6.''
		\item ``We then compute the following equation: ...''
	\end{itemize}
\item Passive Voice:
\begin{itemize}
\item ``The results are shown in Table 6.''
\item ``The following equation is then computed: ...''
\end{itemize}
\end{itemize}

\subsection{Person}
Do not use the first person singular (``I'') in an academic paper. Instead, use ``we'' or use the passive voice.

\section{Regional Differences}
Unless otherwise stated, you are free to choose between American and British English. Once you have done so, stay consistent and stick to the chosen style. Set your spell-check accordingly. Typical spelling differences between American and British English include:

\begin{itemize}
	\item ``o'' vs. ``ou'': \textit{behavior} (Am.) vs. \textit{behaviour} (Br.)
	\item ``er'' vs. ``re'': \textit{center} (Am.) vs. \textit{centre} (Br.)
	\item ``z'' vs. ``s'': \textit{analyzed} (Am.) vs. \textit{analysed} (Br.)
	\item double consonants: \textit{modeled, modeling} (Am.) vs. \textit{modelled, modelling} (Br.) 
\end{itemize}

Reliable dictionaries are Merriam Webster\footnote{https://www.merriam-webster.com} (American) and the Oxford English Dictionary\footnote{https://en.oxforddictionaries.com} (British).